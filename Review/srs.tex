\documentclass{scrreprt}
\usepackage{listings}
\usepackage{underscore}
\usepackage[bookmarks=true]{hyperref}
\usepackage[utf8]{inputenc}
\usepackage[ampersand]{easylist}
\usepackage{enumitem}
\usepackage{graphicx}
\setlist[enumerate]{label*=\arabic*.}

\def\myversion{1.0 }
\title{%
\flushright
\rule{16cm}{5pt}\vskip1cm
\Huge{CODE REVIEW\\ by Team 7}\\
\vspace{2cm}
for\\
\vspace{2cm}
ESE 2014 Project by Team 2\\
\vspace{2cm}
\LARGE{Version \myversion \\}
\vspace{2cm}
S. Berger, J. Niklaus, J. Schaerer, A. Sellathurai \\
\vfill
\rule{16cm}{5pt}
}
\date{}
\usepackage{hyperref}




\begin{document}
\maketitle
\tableofcontents

\chapter{General Review}
\section{Design}
\subsection{Violation of MVC pattern}
Method enlistad in AdController is huge. Sure more could be extracted that does not belong to the controller. Method getUserImage in AdController deals with images as well as enlistad. Maybe a PictureService is missing. PICTURE_LOCATION and servletContext belong to the PictureService as well. The same applies for enlistroomie in RoomieController. Why does does not the Adservice return directly the interessents, but they are constructed in besichtigungsterminSetzen in AppointmentController?

\subsection{Usage of helper objects between view and model}
Method enlistad in AdController uses helper object PictureManager.

\subsection{Rich OO domain model}
What is meant with that?

\subsection{Clear responsibilities}
All success messages collected in success.jsp: They don't have anything in common. Why not in individual pages? Redirects necessary but why? (user cannot interact with website during redirect.) Good use of ErrorSaver object.

\subsection{Sound invariants}
I did not find invariants in the code.

\subsection{Overall code organization and reuse, e.g. views}
exceptions folder and thus InvalidUserExceptions exists twice. But otherwise reuse and organization fine.

\section{Coding Style}
[TBA]

\section{Documentation}
\subsection{Understandable}
In general, the documentation is very understandable. It creates a clear idea of what the code does but is still not too long. Still there are some minor issues to be noticed. An example is AdController.showAdId: The comment isn't very understandable. It talks about "needed criteria" but it is not clear what these are. Similarly, the doc comment for FilterAdsController.filterAdsIndex talks about a "smaller filter" but it is not obvious what this is and why it is needed. Another problematic documentation is the one for AdController.placeYourself. It says it redirects to the "place yourself" page. Unfortunately, it is not clear what this page is or does. As far as it goes for the method name and request mapping, it looks like it is responsible for placing a user, room mate or something like that. This wouldn't really make sense in that class, so it remains unclear what the method exactly does.

\subsection{Intention-revealing}
Throughout the documentation, the intention of each method is clearly shown. The comments give a brief but revealing overview about what the methods are used for, which is going to be a big help for everyone to maintain the code in the future.

\subsection{Describe Responsibilities}
Many parts of the documentation lack a clear description of responsibilities. Often the just say "...is triggered, when...". Sometimes the comments even state which element on which .jsp triggers the method. While this is really helpful when maintaining the code, it is not really the idea of responsibility driven design. If there is going to be another view to use the same method, the documentation will no longer be fully correct. An even worse case could be when a part of the system is changed and the method is not even invoked by the stated view any more. Nevertheless, most responsibilities are still clear, since they're implied by the method name and the commentary. 

\subsection{Consistent Domain Vocabulary}
Most of the used vocabulary seems to be consistent throughout the documentation. This is, with one exception: the term for a person to look for an apartment. It is referred to as "mate", "yourself", "interessent", "interested user", "applicant", "person" and "roomie". There should be one single, exactly defined word for that. Apart from that, the vocabulary is very accurate and adds to quite understandable documentation.

\section{Tests}
[TBA]

\chapter{Code Analysis: AdController}
The AdController class has quite a lot of responsibilities. Most of them make sense as a part of that class since there are a lot of different tasks and views related to ads but there could be a way to split the responsibilities more clearly. There could be made a distinction between displaying and managing ads. This would mean to move methods like "showAds", "showUnfiltredAds" and "showAdId" to a class like "AdDisplayController" and methods like "createAd" and "enlistad" to a one like "AdManagementController". Like this, the responsibilities of the AdController would be split up and could be more clearly defined. \\

Another questionable point is the placeYourself method. It's not exactly clear from method name and documentation what the method does, this is, what the "placeyourself" page is. Looking at method name and request mapping, as well as placeyourself.jsp, it seems to be a page where a user or room mate can somehow be added. If this is the case, then that method should be moved to another class, such as UserController or RoomieController, since it doesn't seem to be in direct relation to ads.

\end{document}